\chapter{Chapter 2 Software Installation}
 
%-------------------------------------------------------------------------------
\section{Introduction}
%-------------------------------------------------------------------------------
\section{Obtaining the FV3SAR Source Code} \label{obtain_src}
The source code for the FV3SAR pre-processing utilities and the regional model reside in two separate 
NOAA VLAB repositories.  The pre-processing utilities are located in the fv3gfs repository.  To clone
this repository, create a directory called \verb|${TOP_DIR}|, clone the repository and check out the 
\verb|regional_UFS| branch:

    \begin{itemize}
       \item[] \verb|% mkdir ${TOP_DIR}|
       \item[] \verb|% cd  ${TOP_DIR}|
       \item[] \verb|% git clone --recursive ssh://${USER}@vlab.ncep.noaa.gov:29418/fv3gfs|
       \item[] \verb|% cd fv3gfs|
       \item[] \verb|% git checkout regional_UFS|
    \end{itemize}

The model source code is located in the NEMSfv3gfs repository, which includes three submodules: FMS, FV3,
and NEMS.  You will need to clone the main repository, checkout the regional\_fv3\_nemsfv3gfs branch, and
check out the appropriate branch for each submodule:

    \begin{enumerate}
       \item[] \verb|% cd ${TOP_DIR}|
       \item[] \verb|% git clone --recursive ssh://${USER}@vlab.ncep.noaa.gov:29418/NEMSfv3gfs|
       \item[] \verb|% cd NEMSfv3gfs|
       \item[] \verb|% git checkout regional_fv3_nemsfv3gfs|
       \item[] \verb|% cd FV3|
       \item[] \verb|% git checkout regional_fv3|
       \item[] \verb|% cd ${TOP_DIR}|
       \item[] \verb|% cd FMS|
       \item[] \verb|% git checkout GFS-FMS|
       \item[] \verb|% cd ${TOP_DIR}|
       \item[] \verb|% cd NEMS|
       \item[] \verb|% git checkout master|
    \end{enumerate}

\section{System Requirements, Libraries and Tools}

\subsection{Supported Platforms and Compilers}

The FV3SAR model is supported on the NOAA HPC Theia and NCAR Supercomputer Cheyenne.  Intel is the only
currently supported compiler for building the pre-processing utilities and the FV3SAR model.

\subsection{NCEP Libraries}

A number of the NCEP (National Center for Environmental Prediction) production libraries are necessary
for building and running the FV3SAR pre-processing utilities and model (Table~\ref{tab:ncep_libs}).
These libraries are not part of the source code distribution.
If they are not already installed on your computer platform, you may have to download some or all the
source code from \url{http://www.nco.ncep.noaa.gov/pmb/codes/nwprod/} and build the libraries yourself.
Note that these libraries must be built with the same compiler used to build the pre-processing utilities
FV3SAR model.
Another option is to clone the git repository \url{https://github.com/climbfuji/NCEPlibs.git} and follow
the build instructions.  This currently includes only the libraries used by the FV3SAR model.

\begin{table}[!htb]
\begin{center}
\begin{tabular}{ l c c }
\hline
 NCEP Library & Pre-Processing & FV3SAR \\ 
\hline
 bacio/v2.0.1       &   & X \\ 
 bacio/v2.0.2       & X &   \\ 
 gfsio/v1.1.0       & X &   \\ 
 ip/v2.0.0          & X & X \\  
 ip/v3.0.0          & X &   \\  
 landsfcutil/v2.1.0 & X &   \\ 
 nemsio/v2.2.3      & X & X \\ 
 nemsiogfs/v2.0.1   & X &   \\ 
 sfcio/v1.0.0       & X &   \\ 
 sigio/v2.0.1       & X &   \\ 
 sp/v2.0.2          & X & X \\ 
 w3emc/v2.0.5       & X & X \\   
 w3emc/v2.2.0       & X &   \\   
 w3nco/v2.0.6       & X & X \\
\hline
\end{tabular}
\caption{\label{tab:ncep_libs}NCEP libraries necessary to build the FV3SAR pre-processing utilities and the model.
                              X indicates that the library is required.}
\end{center}
\end{table}

\subsection{External Libraries}

In addition to the NCEP libraries, several external support libraries are required but not included with
the source code.  Most of these libraries are installed as part of the compiler installation.  For FV3SAR, these
libraries are:

  \begin{itemize}
    \item Intel compiler
    \item impi|
    \item ESMFv7.1.0
    \item netCDF
    \item HDF5
    \item pnetCDF
  \end{itemize}


\section{Compiling the FV3SAR Source Code}

To run the end-to-end FV3SAR forecasting system, the pre-processing utilities, the FV3SAR model, and the
post-porcessing components must be built.  This section describes the steps for the supported compilers
on the available platforms.  The directory \verb|${TOP_DIR}| is assumed to be where the code has been checked 
out as described in Section \ref{obtain_src}.

\subsection{Building the FV3SAR Pre-Processing Utilities}

To build the FV3SAR pre-processing utilities:

\begin{itemize}
  \item[] \verb|% cd ${TOP_DIR}/fv3gfs/regional| 
  \item[] \verb|% ./build_regional ${SITE}|
\end{itemize}

where site is one of "theia", "cheyenne", "wcoss\_cray", or "odin".  When the build completes, you will see 9
executables in \verb|${TOP_DIR}/fv3gfs/exec|:

\begin{itemize}
  \item[] \verb|% ls ${TOP_DIR}/fv3gfs/exec| 
  \item[] \verb|filter_topo        global_chgres         make_solo_mosaic    | 
  \item[] \verb|fregrid            make_hgrid            ml01rg2.x           | 
  \item[] \verb|fregrid_parallel   make_hgrid_parallel   shave.x             | 
\end{itemize}

\subsection{Building the FV3SAR Model}

To build the FV3SAR model:

\begin{itemize}
  \item[] \verb|% cd ${TOP_DIR}/NEMSfv3gfs/tests| 
  \item[] \verb|% ./compile.sh ${TOP_DIR}/NEMSfv3gfs/FV3 ${BUILD_TARGET} >& make.out.32bit|
\end{itemize}

where \verb|${BUILD_TARGET}| is one of

\begin{itemize}
  \item[] \verb| cheyenne.gnu cheyenne.intel cheyenne.pgi  odin |
  \item[] \verb| theia.gnu    theia.intel    theia.pgi     wcoss_cray | 
\end{itemize}

When the regional model compilation is complete, the line

\verb|Elapsed time 568 seconds. Compiling  finished|
 
will be printed at the end of the file.  There will be an executable \verb|NEMS.x| created in
\verb|${TOP_DIR}/NEMSfv3gfs/NEMS/exe/| which will be copied over to \verb|${TOP_DIR}/NEMSfv3gfs/tests/fv3.exe/|.

There are other command-line options available when running \verb|compile.sh| and are shown in Table
\ref{tab:compile_args}.

\begin{table}[!htb]
\begin{center}
\begin{tabular}{ l l l l }
\hline
 Argument Number & Argument Name          & Example                      & Optional \\
\hline
 1               & path to FV3 directory  & ../FV3                       & No       \\ 
 2               & BUILD\_TARGET          & theia.intel, cheyenne.intel  & No       \\ 
 3               & MAKE\_OPT              & DEBUG=Y 32BIT=Y REPRO=N      & Yes      \\ 
 4               & BUILD\_NAME            & 32bit                        & Yes      \\ 
 5               & clean\_before          & YES                          & Yes      \\ 
 6               & clean\_after           & YES                          & Yes      \\ 
\hline
\end{tabular}
\caption{\label{tab:compile_args} Command-line options to build FV3SAR.}
\end{center}
\end{table}

To build with debugging flags, add the following quantities to the \verb|compile.sh| command:

\begin{itemize}
  \item[] \verb|% ./compile.sh ../FV3 ${BUILD_TARGET} "32BIT=Y DEBUG=Y" 32bit \\|
  \item[] \verb| [clean_before] [clean_after] >& make.out.32bit|
\end{itemize}

The last two optional arguments [clean\_before] and [clean\_after] control whether or not to run
make clean to remove temporary files. The default values are “YES”. Specifying “NO” will skip
cleaning step, which will speed up repeating compilation, which is useful for debugging.

Currently all the fixed fields necessary to run a uniform global case without a nest are in subdirectories
on each supported machine:

\begin{itemize}
  \item[] \verb|/gpfs/hps3/emc/global/noscrub/emc.glopara/git/fv3gfs/fix/fix_fv3| on the cray
  \item[] \verb|/scratch4/NCEPDEV/global/save/glopara/git/fv3gfs/fix/fix_fv3| on theia
  \item[] \verb|/glade/p/ral/jntp/GMTB/FV3GFS_V1_RELEASE/fix/fix_am/| on Cheyenne
\end{itemize}

