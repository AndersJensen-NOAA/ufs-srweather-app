\chapter{Community SAR-FV3 Workflow}
 
%-------------------------------------------------------------------------------
\section{Overview}
\begin{itemize}
\item The Community SAR-FV3 workflow and associated scripts are available in the \verb|regional_UFS| branch of the fv3gfs repository.
\item Within the regional directory, a complete workflow for Theia exists (\verb|FV3_Theia.xml|), while a parallel workflow for Cheyenne is currently under development (\verb|FV3_Cheyenne.xml|).  Eventually both workflows will be merged to allow for multi-platform flexibility.
\item The workflow and scripts are designed to accommodate any domain, resolution, stretch factor, \verb|BC_interval|, or forecast length, with the scripts handling all subsequent grid decomposition automatically (only \verb|layout_x|, \verb|layout_y|, and \verb|ncores_per_node| are required as environment variables).
\item All scripts called by the workflow can be found in the ush directory and are run in this order:
    \begin{itemize}
    \item \verb|fv3gfs_driver_grid_regional.sh|
    \item \verb|get_GFS_anl_fcst_files.sh|
    \item \verb|run_chgres_rgnl_IC_BC0.sh|
    \item \verb|run_chgres_rgnl_BCs.sh|
    \item \verb|stage.sh|
    \item \verb|run_FV3.sh|
    \end{itemize}
\item Dependencies exist on successful completion of the previous script/task before the workflow can continue.
\item The stage task in the workflow automatically handles grid decomposition definitions necessary within the input.nml file based on user-defined \verb|layout_x| and \verb|layout_y| and will warn the user (then fail) if the chosen \verb|layout_x| and \verb|layout_y| are incompatible with the generated domain.
\item The shell scripts called by the Rocoto workflow can also be run on the command line without any modifications (in this case, the shell scripts will source a global config.sh file located in ush instead of obtaining environment variable info from the Rocoto workflow)
\end{itemize}

\subsection{Description of Rocoto tasks}

\begin{itemize}
\item \verb|fv3gfs_driver_grid_regional|
    \begin{itemize}
    \item Generates the netCDF files describing the grid and (unfiltered and filtered) orography.  These consist of two grid description files and two filtered orography files.  The first grid file describes a regional grid having a halo region of 3 cells, while the other describes a grid with 4 halo cells.  These two files are obtained by ''shaving'' a file describing a grid with an even larger halo, e.g. 6 cells (which is also output in a temp directory)  Similarly, the first filtered orography file contains 0 halo cells while the second contains 4 halo cells.  These are also obtained by shaving a filtered orography file with an even larger halo (e.g. 6 cells).
    \end{itemize}
\item \verb|get_GFS_anl_fcst_files|
    \begin{itemize}
    \item Copies GFS analysis and forecast files (in nemsio format) that are needed to generate the IC/BC netcdf files for a SAR FV3 run.  These are either copied from disk (if available) or extracted from an archive (tar) file that is copied over from HPSS.  There are 3 analysis files (\verb|gfs.t00z.atmanl.nemsio|, \verb|gfs.t00z.nstanl.nemsio|, and \verb|gfs.t00z.sfcanl.nemsio|) and one forecast file for each hour after the initial time for which a netCDF BC file needs to be generated (e.g. \verb|gfs.t00z.atmf003.nemsio|, \verb|gfs.t00z.atmf006.nemsio|, etc). 
    \end{itemize}
\item \verb|run_chgres_rgnl_IC_BC0|
    \begin{itemize}
    \item Generates the netCDF ICs file and the first (i.e. forecast hour 0) netCDF BCs file.  The ICs file is named gfs\_data.tile7.nc, and the first BCs file is named \verb|gfs_bndy.tile7.000.nc|.
    \end{itemize}
\item \verb|run_chgres_rgn_BCs|
    \begin{itemize}
    \item Generates the netCDF BCs files for all forecast hours AFTER hour 0 at a frequency of \verb|"BC_INTERVAL_HRS"| hours out to \verb|"FCST_LEN_HRS"| (as defined in Rocoto workflow).  These files are named, for example, \verb|gfs_bndy.tile7.003.nc|, \verb|gfs_bndy.tile7.006.nc|, etc.
    \end{itemize}
\item \verb|stage|
    \begin{itemize}
    \item Creates the run directory and copies/links necessary netCDF files generated during the preprocessing tasks into the INPUT directory
    \item Copies template namelist and configure files into the run directory and modifies them according to variables defined in the workflow related to start/end time, forecast length, boundary condition interval, grid decomposition and script-calculated nodes and ppn, etc.
    \item Modifies the Rocoto workflow resource definitions for nodes and processes per node for the run\_FV3 task based on user-defined layout\_x and layout\_y 
\end{itemize}
\item \verb|run_FV3|
    \begin{itemize}
    \item Starts integration of the model through execution of the run.regional script in the run directory
    \end{itemize}
\end{itemize}

This is a dummy citation to stop latex from complaining that the document contains no citations: \cite{Anderson2007}.  
Bye.


%-------------------------------------------------------------------------------
\subsection{Steps to run rocoto tasks}

\begin{enumerate}
\item Set environment variables, \verb|layout_x|, \verb|layout_y|, and \verb|ncores_per_node|
\item Modify ENTITY in \verb ''regional/FV3_Theia.xml'' |, for examples, 
\verb|USER, ACCOUNT, BASEDIR .... RES, FCST_LEN_HRS, BC_INTERVAL_HRS|
\item Build NEMSfv3gfs
\item Launch rocotorun as
\verb|$>rocotorun -w ${pathnam}/FV3_Theia.xml -d ${pathnam}/FV3_Theia.db -v 10|
Where \verb|${pathnam}| is the working directory of fv3gfs. File \verb|?FV3_Theia.db?| need not exist prior to the first time the command is run.
\end{enumerate}



